\documentclass[conference]{IEEEtran}
\IEEEoverridecommandlockouts
% The preceding line is only needed to identify funding in the first footnote. If that is unneeded, please comment it out.
\usepackage{cite}
\usepackage{amsmath,amssymb,amsfonts}
\usepackage{algorithmic}
\usepackage{graphicx}
\usepackage{textcomp}
\usepackage{xcolor}
\usepackage{geometry}

\def\BibTeX{{\rm B\kern-.05em{\sc i\kern-.025em b}\kern-.08em
    T\kern-.1667em\lower.7ex\hbox{E}\kern-.125emX}}
\begin{document}
\title{
A Study Into The Applications of Machine Learning in The Prognostication of Breast Cancer\\
\vspace{5mm}
\large Department of Computing \\
\vspace{3mm} 
\large Letterkenny Institute of Technology \\
\vspace{3mm} 
\large Machine Learning
}
\vspace{3mm}  
\author{Ultan Kearns \& Liam Millar}
\maketitle
\begin{abstract}
    Machine Learning has been used in a variety of healthcare fields and has proven to be very successful in many areas from identifying similarities in different strains of DNA to predicting the prognosis of many types of cancers.  Machine Learning is particularly useful when it comes to analyzing and comparing data, this data can then be used to train a model to notice certain similarities in malignant and benign tumours.  This paper aims to analyze which methods would be best suited to identifying the prognosis of breast cancer.\\
    keywords - Machine Learning, Breast Cancer
\end{abstract}

\section{Introduction}
Machine Learning is a field which aims to train machines to think and learn like humans.  This is done by feeding data into the machine, the machine then analyzes the data and begins to start noticing patterns within the data so that when new data is used the machine can easily detect certain characteristics about the data and classify it.  There are many ways which a machine may choose to learn we will explain the types of learning we used when trying this model in the upcoming sections.  
\subsection{Explain about breast cancer \& prognosis cite statistics}
\subsection{Introduction to machine learning methods and what machine learning is}
\subsection{talk about expert systems?}
\subsection{Discuss various uses of machine learning in medicine today}
\section{Methodology}
\subsection{Describe what we did}
\subsection{Describe algorithms and methods we used}
\subsection{Talk about the }
\section{Applications}
\subsection{discuss future of ML in healthcare}
\subsection{Talk about applications of our model}
\subsection{Talk about various improvements which may be made in future}
\section{Conclusion}
\bibliography{references}

\end{document}
